\documentclass{article}
\usepackage[utf8]{inputenc}
\usepackage{amsmath}
\usepackage{hyperref}
\usepackage{graphicx}
\usepackage{tikz-cd}
\usepackage{amsthm}
\usepackage{csquotes}

\theoremstyle{definition}
\newtheorem{definition}{Definition}[section]
\newtheorem{theorem}{Theorem}
\newtheorem{proposition}{Proposition}
\newtheorem{lemma}{Lemma}
\tikzcdset{row sep/normal=2.7em}
\tikzcdset{column sep/normal=3.6em}

% Helpful command for typing arrows
\newcommand{\arr}[3]{#1 : #2 \rightarrow #3}
\newcommand\id{\mathit{id}}

\title{Category Theory Notes}
\author{}

\begin{document}

\maketitle

\section{What is a category}

\subsection{Definition of a category}
 
\subsection{Examples of categories}

\begin{itemize}
  \item \textbf{Poset}. A poset (partially ordered set) is a set $P$ with a binary
    relation $\leq$ that satisfies reflexivity, antisymmetry, and transivity.
    Any poset forms a category where the elements of the category are the
    objects of $P$ and where for two objects $a$ and $b$ there is an arrow from
    $a$ to $b$ iff. $a \leq b$.
\end{itemize}

\section{Commutative diagrams}

In category theory equations can be expressed using commutative diagrams. When
we draw a diagram like the following.
\[
\begin{tikzcd}
  A \arrow[r, "f"] \arrow[d, "h"']
  & B \arrow[d, "g"] \\
  C \arrow[r, "p"] & D
\end{tikzcd}
\]
It is supposed to communicate that $f \circ g = h \circ p$. In such cases we say
the ``the diagram commutes'' to denote the properties that the diagram indicates
that the arrows have.

As another example, this diagram
\[
  \begin{tikzcd}
    A \arrow[r, "f"] \arrow[dr, "h"] & B \arrow[d, "g"] \\
                                     & C
  \end{tikzcd}
\]
\emph{commutes} if $f \circ g = h$.

In general it is the case that for a commutative diagram all paths with the same
origin and end are equal.

\section{Monomorphism and epimorphism}

\subsection{Monomorphism}

\begin{definition}{Monomorphism}
  In a category $\bf{C}$, an arrow $\arr{f}{A}{B}$ is a \emph{monomorphism} if for
  any $g, h : C \rightarrow A$ it is the case that $fg = fh \implies g = h$.
  \[
  \begin{tikzcd}
    C \ar[r, shift left, "g"] \ar[r, shift right, "h"'] & A \arrow[r, "f"] & B
  \end{tikzcd}
  \]
\end{definition}

A monomorphism is also called a \emph{mono} and it is said to be \emph{monic}.

Mnemonic: A monomorphism is the categorical equivalent to an \emph{injective}
function, that is a one-to-one function, and \emph{mono} means \emph{one}.

\begin{proposition}
  A function $\arr{f}{A}{B}$ between sets is monic in $\bf{Sets}$ if and only if
  it is injective.
\end{proposition}
\begin{proof}
  ($\Rightarrow$) Assume $f$ is a monomorphism in $\bf{Sets}$. We must show that
  $f$ is an injective function. Let $a, a' \in A$ where $a \neq a'$. We are done
  if we can show that $f(a) \neq f(a')$.
  
  ($\Leftarrow$) Assume $f$ is an injective function. We must show that $f$ is a
  monomorphism.
\end{proof}

\begin{definition}{Epimorphism}
  In a category $\bf{C}$, an arrow $\arr{f}{A}{B}$ is an \emph{epimorphism} if
  for any $\arr{i, j}{B}{D}$ it is the case that $if = jf \implies i = j$.

  \begin{tikzcd}
    A \ar[r, "f"] & B \ar[r, shift left, "i"] \ar[r, shift right, "j"'] & D
  \end{tikzcd}

  An epimorphism is said to be \emph{epic}.
\end{definition}
    
\begin{proposition}
  A function $\arr{f}{A}{B}$ between sets is an epimorphism in $\bf{Sets}$ if
  and only if it is injective.
\end{proposition}
\begin{proof}
  ($\Rightarrow$) Assume $f$ is an epimorphism. We must show that $f$ is a
  surjective function. Let $b$ be an element in $B$. Let $d, d' \in D$ where $d
  \neq d'$ and define $\arr{i}{B}{D}$ as the function that maps every element
  in $B$ to $d$ except for $b$ which it maps to $d'$.
  \[
    i(x) = \begin{cases}
      d' & x = b \\
      d & \text{otherwise}
    \end{cases}
  \]
  Let $\arr{j}{B}{D}$ be a constant function that sends every element to $d$.
  \[
    j(x) = d
  \]
  Since $i \neq j$ we have $if \neq jf$. Thus there exists some $a \in A$ where
  $i(f(a)) \neq j(f(a))$. But $i$ and $j$ are equal for all arguments except
  $b$. Hence $f(a) = b$. Since $b$ was arbitrary this shows that $f$ is
  surjective.

  ($\Leftarrow$) Assume that $f$ is a surjective function. We must show that $f$
  is an epimorphism. Let $\arr{i, j}{B}{D}$ be arbitrary functions where $i \neq
  j$. Then there exists some $b \in B$ where $i(b) \neq j(b)$. Since $f$ is
  surjective there must exist an $a \in A$ where $b = f(a)$. Putting this
  together we have:

  \begin{align*}
    i(b) &\neq j(b) && \text{since $i \neq j$} \\
    i(f(a)) &\neq j(f(a)) && \text{from how we defined $b$}
  \end{align*}

  Therefore $if \neq jf$.
\end{proof}

\begin{proposition}
  In a poset category all arrows are both monic and epic.
\end{proposition}
\begin{proof}
  Let $\arr{f}{A}{B}$ be an arrow in a poset category. Per the definition of a
  poset category there is always only a single arrow between two objects.
  Therefore if we have $\arr{g, h}{C}{A}$ such that $fg = fh$ we can conclude
  that $g = h$ since they are both arrows between $C$ and $A$. Therefore $f$ is
  monic. Similarly, if $\arr{i, j}{B}{D}$ such that $if = jf$ then $i = j$ and
  $f$ is a epic.
\end{proof}
 
\begin{proposition}
  In a poset category all arrows are both monic and epic.
\end{proposition}
\begin{proof}
  Let $\arr{f}{A}{B}$ be an arrow in a poset category. Per the definition of a
  poset category there is always only a single arrow between two objects.
  Therefore if we have $\arr{g, h}{C}{A}$ such that $fg = fh$ we can conclude
  that $g = h$ since they are both arrows between $C$ and $A$. Therefore $f$ is
  monic. Similarly, if $\arr{i, j}{B}{D}$ such that $if = jf$ then $i = j$ and
  $f$ is a epic.
\end{proof}

\begin{proposition}
  An isomorphism is both monic and epic.
\end{proposition}
\begin{proof}

\end{proof}

\begin{proposition}[Inverses are unique]
  If an arrow $\arr{f}{A}{B}$ has two inverses $\arr{g, g'}{B}{A}$ then $g =
  g'$.
\end{proposition}
\begin{proof}
  Consider the commutative diagram below.
  \[
    \begin{tikzcd}
      B \arrow[r, "id_B"] \arrow[d, "g"'] & B \arrow[d, "g'"] \\
      A \arrow[r, "id_A"'] \arrow[ru, "f"] & A
    \end{tikzcd}
  \]
  From the diagram we can derive the following equality.
  \[
    g' = g' \id_B = g' (f g) = (g' f) g = \id_A g = g
  \]
\end{proof}

\section{Products}

\begin{lemma}
  Products are unique up to isomorphism. That is, if $A \times B$ and $A' \times
  B'$ are products in a category $\bf{C}$ then.
  $$
  A \times B \cong A' \times B'
  $$
\end{lemma}

The above theorem tells us that all products are the same except for unimportant
details.

\end{document}
