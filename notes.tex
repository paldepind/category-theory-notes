\documentclass{article}
\usepackage[utf8]{inputenc}
\usepackage{amsmath}
\usepackage{hyperref}
\usepackage{graphicx}
\usepackage{tikz-cd}
\usepackage{amsthm}
\usepackage{csquotes}

\theoremstyle{definition}
\newtheorem{definition}{Definition}[section]
\newtheorem{theorem}{Theorem}
\newtheorem{proposition}{Proposition}
\newtheorem{lemma}{Lemma}
\tikzcdset{row sep/normal=2.7em}
\tikzcdset{column sep/normal=3.6em}

% Helpful command for typing arrows
\newcommand{\arr}[3]{#1 : #2 \rightarrow #3}
\newcommand\id{\mathit{id}}

\title{Category Theory Notes}
\author{}

\begin{document}

\maketitle

\section{What is a category theory}

``As a first approximation category theory is the abstract algebra of
functions.'' - Steve Awodey

\subsection{Definition of a category}

objects, A, B, C, ...
arrows f, g, h, ...
each function have a domain $dom(f)$ and codomain $cod(f)$
for each composable pair of arrows $\arr{f}{A}{B}$ and $\arr{g}{B}{C}$.

Laws.

\begin{itemize}
  \item Unit
  \item Associativity
\end{itemize}

\subsection{Examples of categories}

\begin{itemize}
  \item \textbf{Sets}. The objects are sets and the arrows are functions.
  \item \textbf{1}. The category with one object and one arrow. It looks like this.
    \[
      \begin{tikzcd}
        \bullet \arrow[loop right]
      \end{tikzcd}
    \]
  \item A poset $(P, \leq)$ (partially ordered set) is a set $P$ with a binary
    relation $\leq$ that satisfies reflexivity, antisymmetry, and transivity.
    Any poset forms a category where the elements of the category are the
    objects of $P$ and where for two objects $a$ and $b$ there is an arrow from
    $a$ to $b$ iff. $a \leq b$:
    \[
      p \rightarrow q \iff p \leq q
    \]
    A category based on a poset is called a \emph{poset category}.

    Note that \textbf{Poset} is a category with ``few'' arrows. Between two
    objects $p$ and $q$ there is at most one arrow.

  \item \textbf{Poset} (or \textbf{Pos}) The category where the objects are all
    posets and the arrows are monotone functions.

  \item Any monoid forms a category. The category contains a single object. Each
    element in the monoid is an arrow in the category and composition of arrows
    is the composition from the monoid. The identity arrow is the identity
    element in the monoid. This construction satisfies the law of a category
    directly from the laws of a monoid.

    Note that a category based on a monoid is a category with ``few'' objects.
    There is only a single objects. Furthermore a category with one object is
    always a monoid. Thus, \emph{a monoid is just a category with one object}.

  \item \textbf{Monoid} (or \textbf{Mon}) The category where the objects are all
    monoid and the arrows are monoid homomorphisms.

  \item \textbf{Rel}. The objects are sets and the arrows are relations.
  \end{itemize}
\end{itemize}

\section{Commutative diagrams}

In category theory equations can be expressed using commutative diagrams. When
we draw a diagram like the following.
\[
\begin{tikzcd}
  A \arrow[r, "f"] \arrow[d, "h"']
  & B \arrow[d, "g"] \\
  C \arrow[r, "p"] & D
\end{tikzcd}
\]
It is supposed to communicate that $f \circ g = h \circ p$. In such cases we say
the ``the diagram commutes'' to denote the properties that the diagram indicates
that the arrows have.

As another example, this diagram
\[
  \begin{tikzcd}
    A \arrow[r, "f"] \arrow[dr, "h"] & B \arrow[d, "g"] \\
                                     & C
  \end{tikzcd}
\]
\emph{commutes} if $f \circ g = h$.

In general it is the case that for a commutative diagram all paths with the same
origin and end are equal.

\section{Functor}

\begin{definition}[Functor]
  A functor $\arr{F}{\mathbf{C}}{\mathbf{D}}$ is a map between categories. It
  consists of two parts:
  \begin{itemize}
    \item A mapping between objects: $\arr{F_{obj}}{obj(\mathbf{C})}{obj\mathbf{D}}$.
    \item A mapping between arrows: $\arr{F_{arr}}{arr(\mathbf{C})}{arr\mathbf{D}}$.
  \end{itemize}
  A functor must satisfy laws:
  \begin{itemize}
    \item Preserve domain and codomain.
    \item Preserve identities. $F(\id_A) = \id_{F(A)}$
    \item Preserve composition: $F(g \circ f) = F(g) \circ F(f)$
  \end{itemize}
\end{definition}

\section{Isomorphism}

\begin{definition}[Isomorphism]
  An isomorphism
\end{definition}

Examples.
\begin{itemize}
  \item In a monoid category the isomorphisms are the elements that have an inverse.
  \item in a group every element has an inverse. Therefore, in a group every
    arrow is and isomorphism.
\end{itemize}

\section{Monomorphism and epimorphism}

\subsection{Monomorphism}

\begin{definition}{Monomorphism}
  In a category $\bf{C}$, an arrow $\arr{f}{A}{B}$ is a \emph{monomorphism} if for
  any $g, h : C \rightarrow A$ it is the case that $fg = fh \implies g = h$.
  \[
  \begin{tikzcd}
    C \ar[r, shift left, "g"] \ar[r, shift right, "h"'] & A \arrow[r, "f"] & B
  \end{tikzcd}
  \]
\end{definition}

A monomorphism is also called a \emph{mono} and it is said to be \emph{monic}.

Mnemonic: A monomorphism is the categorical equivalent to an \emph{injective}
function, that is a one-to-one function, and \emph{mono} means \emph{one}.

\begin{proposition}
  A function $\arr{f}{A}{B}$ between sets is monic in $\bf{Sets}$ if and only if
  it is injective.
\end{proposition}
\begin{proof}
  ($\Rightarrow$) Assume $f$ is a monomorphism in $\bf{Sets}$. We must show that
  $f$ is an injective function. Let $a, a' \in A$ where $a \neq a'$. We are done
  if we can show that $f(a) \neq f(a')$.

  ($\Leftarrow$) Assume $f$ is an injective function. We must show that $f$ is a
  monomorphism.
\end{proof}

\begin{definition}{Epimorphism}
  In a category $\bf{C}$, an arrow $\arr{f}{A}{B}$ is an \emph{epimorphism} if
  for any $\arr{i, j}{B}{D}$ it is the case that $if = jf \implies i = j$.

  \begin{tikzcd}
    A \ar[r, "f"] & B \ar[r, shift left, "i"] \ar[r, shift right, "j"'] & D
  \end{tikzcd}

  An epimorphism is said to be \emph{epic}.
\end{definition}

\begin{proposition}
  A function $\arr{f}{A}{B}$ between sets is an epimorphism in $\bf{Sets}$ if
  and only if it is injective.
\end{proposition}
\begin{proof}
  ($\Rightarrow$) Assume $f$ is an epimorphism. We must show that $f$ is a
  surjective function. Let $b$ be an element in $B$. Let $d, d' \in D$ where $d
  \neq d'$ and define $\arr{i}{B}{D}$ as the function that maps every element
  in $B$ to $d$ except for $b$ which it maps to $d'$.
  \[
    i(x) = \begin{cases}
      d' & x = b \\
      d & \text{otherwise}
    \end{cases}
  \]
  Let $\arr{j}{B}{D}$ be a constant function that sends every element to $d$.
  \[
    j(x) = d
  \]
  Since $i \neq j$ we have $if \neq jf$. Thus there exists some $a \in A$ where
  $i(f(a)) \neq j(f(a))$. But $i$ and $j$ are equal for all arguments except
  $b$. Hence $f(a) = b$. Since $b$ was arbitrary this shows that $f$ is
  surjective.

  ($\Leftarrow$) Assume that $f$ is a surjective function. We must show that $f$
  is an epimorphism. Let $\arr{i, j}{B}{D}$ be arbitrary functions where $i \neq
  j$. Then there exists some $b \in B$ where $i(b) \neq j(b)$. Since $f$ is
  surjective there must exist an $a \in A$ where $b = f(a)$. Putting this
  together we have:

  \begin{align*}
    i(b) &\neq j(b) && \text{since $i \neq j$} \\
    i(f(a)) &\neq j(f(a)) && \text{from how we defined $b$}
  \end{align*}

  Therefore $if \neq jf$.
\end{proof}

\begin{proposition}
  In a poset category all arrows are both monic and epic.
\end{proposition}
\begin{proof}
  Let $\arr{f}{A}{B}$ be an arrow in a poset category. Per the definition of a
  poset category there is always only a single arrow between two objects.
  Therefore if we have $\arr{g, h}{C}{A}$ such that $fg = fh$ we can conclude
  that $g = h$ since they are both arrows between $C$ and $A$. Therefore $f$ is
  monic. Similarly, if $\arr{i, j}{B}{D}$ such that $if = jf$ then $i = j$ and
  $f$ is a epic.
\end{proof}

\begin{proposition}
  In a poset category all arrows are both monic and epic.
\end{proposition}
\begin{proof}
  Let $\arr{f}{A}{B}$ be an arrow in a poset category. Per the definition of a
  poset category there is always only a single arrow between two objects.
  Therefore if we have $\arr{g, h}{C}{A}$ such that $fg = fh$ we can conclude
  that $g = h$ since they are both arrows between $C$ and $A$. Therefore $f$ is
  monic. Similarly, if $\arr{i, j}{B}{D}$ such that $if = jf$ then $i = j$ and
  $f$ is a epic.
\end{proof}

\begin{proposition}
  An isomorphism is both monic and epic.
\end{proposition}
\begin{proof}

\end{proof}

\begin{proposition}[Inverses are unique]
  If an arrow $\arr{f}{A}{B}$ has two inverses $\arr{g, g'}{B}{A}$ then $g =
  g'$.
\end{proposition}
\begin{proof}
  Consider the commutative diagram below.
  \[
    \begin{tikzcd}
      B \arrow[r, "id_B"] \arrow[d, "g"'] & B \arrow[d, "g'"] \\
      A \arrow[r, "id_A"'] \arrow[ru, "f"] & A
    \end{tikzcd}
  \]
  From the diagram we can derive the following equality.
  \[
    g' = g' \id_B = g' (f g) = (g' f) g = \id_A g = g
  \]
\end{proof}

\section{Products}

\begin{lemma}
  Products are unique up to isomorphism. That is, if $A \times B$ and $A' \times
  B'$ are products in a category $\bf{C}$ then.
  $$
  A \times B \cong A' \times B'
  $$
\end{lemma}

The above theorem tells us that all products are the same except for unimportant
details.

\begin{lemma}
  In a category with binary products the binary product operation $A \times B$
  is associative up to isomorphism.
  \[
    (A \times B) \times C \cong a \times (B \times C)
  \]
\end{lemma}
\begin{proof}

\end{proof}

\end{document}
